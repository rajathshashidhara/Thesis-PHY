\chapter{Brillouin-Wigner Perturbation Theory}
When encountered with an analytically intractable quantum mechanical problem, perturbation techniques 
may be used to obtain approximate solutions. Formally, the problem can be stated as follows : 
Given a hamiltonian $\hat{H} = \hat{H}_{0} + \hat{V}$ where $\hat{H}_{0}$ is exactly solvable and 
$\hat{V}$ is the perturbation term, and the eigendecomposition of $\hat{H}_{0}\quad$ \textemdash 
\begin{gather*}
  \hat{H}_{0}\ket{n} = \epsilon_{n}\ket{n} \\
  \braket{m|n} = \delta_{mn} \\
  \sum_{m} {\ket{m} \bra{m}} = \mathrm{1}
\end{gather*}
Brillouin-Wigner (BW) perturbation theory expresses $\{\ket{\psi_n},\dots\}$ and $\{E_n\,,\dots\}$, such that
$\hat{H}\ket{\psi_n} = E_n\ket{\psi_n}$ in terms of $\hat{V}$ and $\{\ket{n}\,,\dots\}$.
A simplistic exposition of BW theory is presented below.

To obtain the BW perturbative expansion, we begin with the eigenvalue equation. 
\begin{equation*}
 \hat{H}\ket{\psi_n} = (\hat{H}_{0} + \hat{V})\ket{\psi_n} = E_n\ket{\psi_n}
\end{equation*}
The wavefunctions $\ket{\psi_n}$ are normalized as $\braket{n|\psi_n} = 1$, as discussed in \cite{sakurai2011modern}. On 
contracting with $\bra{n}$,
\begin{gather}
 \braket{n|(\hat{H}_{0} + \hat{V})| \psi_n} = E_n\braket{n|\psi_n} \nonumber \\
 \epsilon_n \braket{n|\psi_n} + \braket{n|\hat{V}|\psi_n} = E_n\braket{n|\psi_n} \nonumber \\
 \label{app_1:eneryeq} E_n = \epsilon_n + \braket{n|\hat{V}|\psi_n}
\end{gather}
Rewriting the eigenvalue equation as
\begin{align*}
 (E_n - \hat{H}_{0})\ket{\psi_n} &= \hat{V}\ket{\psi_n} \\
 &= \mathrm{1}\; \hat{V}\ket{\psi_n} \\
 &= \sum_{m}{\ket{m}\bra{m}}\; \hat{V}\ket{\psi_n} \\
 &= \ket{n}\braket{n|\hat{V}|\psi_n} + (\mathrm{1} - \ket{n}\bra{n})\hat{V}\ket{\psi_n} \\
 \text{Using Eq. \ref{app_1:eneryeq},} \\
 &= (E_n - \hat{H}_{0})\ket{n} + (\mathrm{1} - \ket{n}\bra{n})\hat{V}\ket{\psi_n} \\
 (E_n - \hat{H}_{0}) (\ket{\psi_n} - \ket{n}) &= (\mathrm{1} - \ket{n}\bra{n})\hat{V}\ket{\psi_n} \\
 \ket{\psi_n} &= \ket{n} + (E_n - \hat{H}_{0})^{-1} (\mathrm{1} - \ket{n}\bra{n})\hat{V}\ket{\psi_n}
\end{align*}
Define resolvent operator as $\hat{R}_{n} = (E_n - \hat{H}_{0})^{-1} = \sum_{n}{\ket{n}(E_n - \epsilon_n)^{-1} \bra{n}}$,
\begin{equation}
  \label{app_1:bweqn}\ket{\psi_n} = \ket{n} +  \hat{R}_{n}(\mathrm{1} - \ket{n}\bra{n})\hat{V}\ket{\psi_n}
\end{equation}
The above iterative equation is the main result of BW perturbation theory. Solving Eq. \eqref{app_1:bweqn} self-consistently with 
Eq. \eqref{app_1:eneryeq}, solutions to the eigenvalue equation are obtained. No approximation has been used until this point, and
exact solution can be obtained if iterated infinitely. In practice, approximate solution is obtained by truncating the iteration.

Further, Eq. \eqref{app_1:bweqn} can be simplified by expanding the recurrence relation
\begin{align}
 \ket{\psi_n} &= \ket{n} + \hat{R}_{n}\hat{Q}_{n}\hat{V}\ket{\psi_n} \nonumber\\
 \text{where } \hat{Q}_{n} = \mathrm{1} - \ket{n}\bra{n}, \nonumber\\
 &= \ket{n} + \hat{R}_{n}\hat{Q}_{n}\hat{V}\ket{n} + \hat{R}_{n}\hat{Q}_{n}\hat{V}\hat{R}_{n}\hat{Q}_{n}\hat{V}\ket{n} + \dots \nonumber\\
 &= \sum_{k=0}^{\infty}{\{\hat{R}_{n}\hat{Q}_{n}\hat{V}\}^k \ket{n}} \nonumber\\ 
 &= (\mathrm{1} - \{\hat{R}_{n}\hat{Q}_{n}\hat{V}\})^{-1} \ket{n}
\end{align}
When higher order term contributions are diminishingly small, truncating the series produces approximate solutions 
to the problem.

Unlike Rayleigh-Schrodinger (RS) perturbation theory, BW theory does not rely on power series expansion requiring strict 
analyticity and does not require seperate treatment of degenerate case. RS theory is an approximation to BW theory 
obtained by power series expansion of $(E_n - \epsilon_m)^{-1}$ in the resolvent operator \cite{wilson2009brillouin}. 
\parencite{silvert1972comparison} provides an excellent comparison between RS and BW perturbation 
theories including situations where BW perturbation technique is more applicable.

Recent efforts to extend the theory to many-body systems has led to systematization of BW theory in 
terms of model space and effective hamiltonian formalism. This new representation requires the
introduction of a model space with respect to a set of reference states. Any complete set of orthonormal states of hilbert space 
is chosen as a set of reference states. Usually, eigenstates of the unperturbed hamiltonian $\hat{H}_{0}$ is chosen as a set of 
reference states. The hilbert space is partitioned into model space and orthogonal space, by choosing one state from the set of
reference states as model state \footnote{In this treatment, a single reference state is chosen as the model function. See 
\cite{wilson2009brillouin} for multi-reference partitioning.}.

Let $\mathrm{R}\equiv\{\ket{\phi_n}\dots\}$ is the set of reference states, $\ket{\phi_0} \in \mathrm{R}$ is the model state, 
then $P = \ket{\phi_0}\bra{\phi_0}$ is the corresponding projection operator of the model space and $Q = \mathrm{1} - P$ is the
projection operator corresponding to orthogonal space. A state $\ket{\psi}$ in the hilbert space can be projected onto the model 
space using operator $P$, $\ket{\phi} = P\ket{\psi}$ and a wavefunction $\ket{\phi}$ in model space can be reconstructed in hilbert
space using the wave operator $\Omega$ as $\ket{\psi} = \Omega\ket{\phi}$. Provided the eigenvalue equation
$\hat{H}\ket{\psi} = E\ket{\psi}$, then $\ket{\phi} = P\ket{\psi}$ satisfies the equation 
$\hat{H}_{eff}\ket{\phi} = E\ket{\phi}$, where 
\begin{equation}
  \hat{H}_{eff} = P\hat{H}\Omega P
\end{equation}
\begin{align*}
 \hat{H}_{eff}\ket{\phi} &= P\hat{H}\Omega P \ket{\phi} \\
 &= P\hat{H}\Omega P P\ket{\psi} \\
 &= P\hat{H}\ket{\psi} \\
 &= E\,P\ket{\psi} \\
 &= E\ket{\phi}
\end{align*}

Therefore, the eigenvalues of the effective hamiltonian are equal to the eigenvalues of the original hamiltonian and the 
eigenfunctions of the original hamiltonian can be obtained by application of the wave operator $\Omega$ on the eigenfunctions of
the effective hamiltonian $\ket{\psi} = \Omega\ket{\phi}$.

How do we obtain the wave operator $\Omega$ ? Operating $Q$ on the eigenvalue equation,
\begin{align*}
 Q\hat{H}\ket{\psi} &= E\,Q\ket{\psi} \\
 Q\ket{\psi} &= \frac{Q\hat{H}}{E}\ket{\psi}
\end{align*}
Therefore,
\begin{align}
  \ket{\psi} &= (P + Q)\,\ket{\psi} \nonumber \\
  &= P\ket{\psi} + Q\ket{\psi} \nonumber \\
  \label{app_1:effform} &= P\ket{\psi} + \frac{Q\hat{H}}{E}\ket{\psi} \\
  P\ket{\psi} &= \left(1-\frac{Q\hat{H}}{E}\right)\ket{\psi} \nonumber \\
  \label{app_1:waveop} \ket{\psi} &= \left(1-\frac{Q\hat{H}}{E}\right)^{-1} P^2 \ket{\psi}  
\end{align} Eq. \eqref{app_1:effform} has the familiar iterative form as Eq. \eqref{app_1:bweqn} with the choice of 
$P = \ket{n}\bra{n}$. From Eq. \eqref{app_1:waveop}, 
\begin{equation}
 \Omega = \left(1-\frac{Q\hat{H}}{E}\right)^{-1}P
\end{equation}
The inverse operation in the wave operator can be expanded to obtain the perturbative expansion.

Above presentation of partitioning and effective hamiltonian formalism of BW theory is described in 
detail in \parencite{wilson2009brillouin}.