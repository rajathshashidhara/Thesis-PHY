% Author	Rajath Shashidhara
% Email		rajath.shashidhara@gmail.com
%
% This work is licensed under the Creative Commons Attribution 4.0 International License. 
% To view a copy of this license, visit http://creativecommons.org/licenses/by/4.0/.

\chapter{Useful Trignometric results} \label{app:trig}
A list of formulas or series expansions that were used in deriving some results obtained in this thesis is given below.
These formulas were obtained from \parencite{abramowitz2012handbook}.
\begin{align}
  e^{i\theta} &= \cos(\theta) + i\sin(\theta) \\
  \cos(A + B) &= \cos(A) \cos(B) - \sin(A)\sin(B) \\
  \sin(A + B) &= \sin(A) \cos(B) + \cos(A)\sin(B) \\
  \cos(r\sin(\theta)) &= \mathcal{J}_{0}(r) + 2\sum_{k=1}^{\infty}\mathcal{J}_{2k}(r) \cos(2k\theta) \\
  \sin(r\sin(\theta)) &= 2\sum_{k=1}^{\infty}\mathcal{J}_{2k-1}(r) \sin((2k-1)\theta) \\
  \cos(r\cos(\theta)) &= \mathcal{J}_{0}(r) + 2\sum_{k=1}^{\infty}(-1)^{k}\mathcal{J}_{2k}(r) \cos(2k\theta) \\
  \sin(r\sin(\theta)) &= 2\sum_{k=1}^{\infty}(-1)^{k-1}\mathcal{J}_{2k-1}(r) \sin((2k-1)\theta)
\end{align}
$\mathcal{J}_{i}(r)$ is Bessel function of the first kind of order $i$.
% 
% \begin{enumerate}
%  \item \begin{equation}
%         e^{i\theta} = \cos(\theta) + i\sin(\theta)
%        \end{equation}
%  \item \begin{equation}
%         \cos(A + B) = \cos(A) \cos(B) - \sin(A)\sin(B)
%        \end{equation}
%  \item \begin{equation}
%         \sin(A + B) = \sin(A) \cos(B) + \cos(A)\sin(B)
%        \end{equation}
%  \item \begin{equation}
%         \cos(r\sin(\theta)) = \mathcal{J}_{0}(r) + 2\sum_{k=1}^{\infty}\mathcal{J}_{2k}(r) \cos(2k\theta)
%        \end{equation}
%  \item \begin{equation}
%         \sin(r\sin(\theta)) = 2\sum_{k=1}^{\infty}\mathcal{J}_{2k-1}(r) \sin((2k-1)\theta)
%        \end{equation}
%  \item \begin{equation}
%         \cos(r\cos(\theta)) = \mathcal{J}_{0}(r) + 2\sum_{k=1}^{\infty}(-1)^{k}\mathcal{J}_{2k}(r) \cos(2k\theta)
%        \end{equation}
%  \item \begin{equation}
%         \sin(r\sin(\theta)) = 2\sum_{k=1}^{\infty}(-1)^{k-1}\mathcal{J}_{2k-1}(r) \sin((2k-1)\theta)
%        \end{equation}
% \end{enumerate}
