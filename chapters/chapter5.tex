% Author	Rajath Shashidhara
% Email		rajath.shashidhara@gmail.com
%
% This work is licensed under the Creative Commons Attribution 4.0 International License. 
% To view a copy of this license, visit http://creativecommons.org/licenses/by/4.0/.

\chapter{Floquet theory}\label{ch:fl}
Although Rayleigh-Schrodinger perturbation theory can be extended to time-dependent situations, it is not well suited for periodic systems as it relies on power-series
expansions. Approximate solutions obtained from RS theory may not retain the periodic character - necessitated by physical conditions - at all perturbation orders \cite{hanggi1998driven}.
A relatively new class of techniques based on the application of floquet theory have been developed to alleviate this problem. This approach effectively converts a time-dependent
problem into a time-independent problem, which is much easier to tackle both analytically and numerically.
\section{Statement}
The time-dependent Schrodinger equation is
\begin{equation}
  \label{chap_5:timeschroeqn}i\hbar\frac{\partial}{\partial t}\ket{\psi (t)} = \hat{H}\ket{\psi (t)}
\end{equation}We are interested in periodically driven systems, i.e., the Hamiltonian satisfies
\begin{equation}
  \hat{H}(t+T) = \hat{H}(t) \quad \forall t
\end{equation} where $T = \frac{2\pi}{\omega}$ is the time period of the driving. Floquet theory establishes the general form of the solution 
to such periodic linear differential equations.
\newline
\begin{theorem}[Floquet Theorem] A periodic linear differential equation 
  \begin{equation*}
  \frac{\partial\vec{x}}{\partial t} = \mathbf{A}(t) \vec{x} \qquad \forall t \; \mathbf{A}(t+T) = \mathbf{A}(t)
  \end{equation*} has solutions of the form 
  \begin{equation*}
    \vec{x} = \mathit{e}^{-i \mu t} \vec{p}(t) \quad \ni \quad \vec{p}(t+T) = \vec{p}(t)
  \end{equation*}
\end{theorem}
Using floquet theory, solutions to Eq. \eqref{chap_5:timeschroeqn} have the form
\begin{equation}
 \label{chap_5:floquetstate}\ket{\psi_{\alpha}(t)} = \mathit{e}^{-i \epsilon_{\alpha} t} \ket{\phi_{\alpha}(t)}
\end{equation} $\ket{\phi_{\alpha}(t)}$ are periodic in time and they are called the \emph{floquet modes}. $\epsilon_{\alpha}$ are real parameters known as \emph{quasienergies}, 
and they are analogous to quasimomentum $k$ in bloch theory. 

\begin{align*}
  \ket{\psi_{\alpha}(t+T)} &= \mathit{e}^{-i \epsilon_{\alpha} (t+T)} \ket{\phi_{\alpha}(t+T)} \\
  &= \mathit{e}^{-i \epsilon_{\alpha} T} \mathit{e}^{-i \epsilon_{\alpha} t} \ket{\phi_{\alpha}(t)} \\  
  &= \mathit{e}^{-i \epsilon_{\alpha} T} \ket{\psi_{\alpha}(t)}
\end{align*}
Three interesting properties can be inferred from the above snippet
\begin{enumerate}
 \item $\epsilon_n$ is real. This is due to the fact that $\ket{\psi_{\alpha}(t)}$ is normalized to $1$ at all times.
 \item If $\hat{U}(t_2, t_1)$ is the time evolution operator, then 
 \begin{equation}
  \label{chap_5:stateseigstrobo}\hat{U}(t + T, t)\ket{\psi_{\alpha}(t)} = \mathit{e}^{-i \epsilon_{\alpha} T} \ket{\psi_{\alpha}(t)}
 \end{equation} Therefore, $\ket{\psi_{\alpha}(t)}$, also known as the \emph{floquet states}, are the eigenvectors of the time evolution operator over one period. 
 This also means that at any fixed time $t$, the floquet states form a complete orthonormal basis.
 \item $\epsilon_\alpha$ may be replaced by $\epsilon_\alpha + n\omega$ without affecting the above equations. We define $\epsilon_{\alpha n} = \epsilon_\alpha + n\omega$ and 
 restrict $\epsilon_\alpha \in \left[\frac{-\omega}{2}, \frac{\omega}{2}\right)$, called the \emph{Brillouin zone}.
\end{enumerate}

The time evolution operator can be further simplified using the complete and orthonormal set of floquet states
\begin{align}
  \hat{U}(t_2, t_1) &= \sum_{\alpha_2}{\ket{\psi_{\alpha_2}(t_2)}\bra{\psi_{\alpha_2}(t_2)}}\hat{U}(t_2, t_1)\sum_{\alpha_1}{\ket{\psi_{\alpha_1}(t_1)}\bra{\psi_{\alpha_1}(t_1)}} \nonumber \\
  &= \sum_{\alpha_1, \alpha_2}{\ket{\psi_{\alpha_2}(t_2)}\braket{\psi_{\alpha_2}(t_2)|\psi_{\alpha_1}(t_2)}\bra{\psi_{\alpha_1}(t_1)}} \nonumber \\
  \label{chap_5:timeevol} &= \sum_{\alpha}{e^{-i\epsilon_\alpha (t_2 - t_1)} \ket{\phi_\alpha (t_2)}\bra{\phi_\alpha (t_1)}}
\end{align} This is a remarkable result as we can express any state $\ket{\psi(t)}$ as
\begin{equation}
  \ket{\psi(t)} = \sum_{\alpha}{\braket{\phi_\alpha (t_0) | \psi(t_0)} e^{-i\epsilon_\alpha (t - t_0)} \ket{\phi_\alpha (t)}} 
\end{equation} i.e., the contribution of each floquet state remains constant through as the state evolves in time.

\section{Floquet Hamiltonian}
The time evolution over one period is given by the \emph{stroboscopic operator} $\hat{U}(t_0 + T, t_0)$. We know from elementary linear algebra that any unitary operator can
be expressed in terms of exponential of a hermitian operator. The generator of stroboscopic evolution, identified by the relation
\begin{equation}
 \exp\left(-iT\hat{H}_{t_0}^{F}\right) = \hat{U}(t_0 + T, t_0)
\end{equation} is known as the \emph{Floquet Hamiltonian}. The floquet Hamiltonian is time-independent, but parameterized by the initial time $t_0$.
It is then obvious from Eq. \eqref{chap_5:stateseigstrobo} that,
\begin{equation*}
 \hat{H}_{t_0}^{F} = \sum_{\alpha}{\epsilon_\alpha \ket{\phi_\alpha (t_0)}\bra{\phi_\alpha (t_0)}}
\end{equation*} and subsequently 
\begin{equation}
 \hat{H}_{t_0}^{F}\ket{\phi_\alpha (t_0)} = \epsilon_\alpha \ket{\phi_\alpha (t_0)}
\end{equation} This eigenvalue equation proves the corollary of floquet theory stating that there exists a one-to-one mapping between the quasienergies and the floquet modes. 
This fact has been inherently assumed in our analysis up to this point, as the quasienergies and floquet modes were indexed by the same variable.

We can obtain an alternate definition of floquet Hamiltonian also known as the \emph{Quasienergy operator} by substitution of Eq. \eqref{chap_5:floquetstate} in 
Eq. \eqref{chap_5:timeschroeqn},
\begin{equation}
  \left[\hat{H} - i\frac{\partial}{\partial t}\right]\ket{\phi_\alpha} = \epsilon_\alpha\ket{\phi_\alpha}
\end{equation} Therefore, we identify $\hat{H}_{t}^{F}$ as 
\begin{equation}
 \hat{H}_{t}^{F} = \left[\hat{H}(t) - i\frac{\partial}{\partial t}\right]
\end{equation}



We also introduce the \emph{micromotion operator} 
\begin{gather}
  \ket{\phi_\alpha (t_2)} = \hat{U}_{F}(t_2, t_1)\ket{\phi_\alpha (t_1)}\\
  \hat{U}_{F}(t_2, t_1) = \sum_{\alpha}{\ket{\phi_\alpha (t_2)}\bra{\phi_\alpha (t_1)}}
\end{gather} which describes the time evolution of periodic floquet modes.

Further, the time evolution operator from Eq. \eqref{chap_5:timeevol} can be written in terms of the floquet Hamiltonian and the micromotion operator.
\begin{equation}
 \hat{U}(t_2, t_1) = e^{-i(t_2 - t_1)\hat{H}_{t_2}^{F}}\hat{U}_{F}(t_2, t_1) = \hat{U}_{F}(t_2, t_1)e^{-i(t_2 - t_1)\hat{H}_{t_1}^{F}} = e^{i\hat{K}(t_2, t_1)}e^{-i(t_1 - t_2)\hat{H}_{t_1}^{F}}
\end{equation} In the final expression, the micromotion operator (unitary operator) is expressed in terms of an exponential.

By diagonalizing the floquet Hamiltonian, both quasienergies and the floquet modes can be determined. Floquet Hamiltonian is amenable to analysis as it is time independent
as opposed to the original time dependent Hamiltonian. Unfortunately, our discussion to this point, does not provide a procedure to calculate the floquet Hamiltonian without
knowing the quasienergies and the floquet modes! A myriad of approximation schemes have been devised to obtain the floquet Hamiltonian for high-frequency driving 
\parencite{casas2001floquet,goldman2014periodically,anisimovas2015high,mikami2016brillouin,grozdanov1988quantum}. We shall only briefly discuss two such methods in the forthcoming sections.

\section{Effective Hamiltonian}
We have already noted that the floquet Hamiltonian is parameterized by initial time. However, this idea of initial time is absurd in many situations.
Instead, a static Hamiltonian without reference to any initial time parameter may be defined without losing the physical interpretation of macromotion associated with the 
floquet Hamiltonian \parencite{rahav2003effective, goldman2014periodically, anisimovas2015high}.

To achieve this, we have to conceive of a unitary transformation $\hat{U}_{F}(t)$, such that 
\begin{equation}
 \label{chap_5:unitarytrans}\hat{H}_{F} = \hat{U}_{F}(t)\hat{H}_{t}^{F}\hat{U}_{F}^{\dagger}(t) = \hat{U}_{F}(t)\hat{H}(t)\hat{U}_{F}^{\dagger}(t) - i\hat{U}_{F}(t)\frac{\partial}{\partial t}\hat{U}_{F}^{\dagger}(t)
\end{equation} is time independent \cite{rahav2003effective, anisimovas2015high}. $\hat{H}_{F}$ is known as the effective Hamiltonian. Under this unitary transformation, quite obviously the floquet modes would transform to
\begin{equation}
 \ket{\phi_\alpha^{F}} = \hat{U}_{F}(t)\ket{\phi_\alpha (t)}
\end{equation} $\ket{\phi_\alpha^{F}}$s are time independent as they are the eigenfunctions of static $\hat{H}_{F}$.
The micromotion operator written in terms of the unitary transformation is
\begin{align}
 \hat{U}_{F}(t_2, t_1) &= \sum_{\alpha}{\ket{\phi_\alpha (t_2)}\bra{\phi_\alpha (t_1)}} \nonumber \\
 &= \sum_{\alpha}{\hat{U}_{F}^{\dagger}(t_2)\ket{\phi_\alpha^{F}}\bra{\phi_\alpha^{F}}\hat{U}_{F}(t_1)} \nonumber \\
 &= \hat{U}_{F}^{\dagger}(t_2)\left(\sum_{\alpha}{\ket{\phi_\alpha^{F}}\bra{\phi_\alpha^{F}}}\right)\hat{U}_{F}(t_1) \nonumber \\
 &= \hat{U}_{F}^{\dagger}(t_2)\hat{U}_{F}(t_1)
\end{align}

In the time-independent gauge, time evolution is simply $e^{-i\hat{H}_{F}(t_2 - t_1)}$. To propagate any state in the time-independent gauge, it has to be transformed to 
time-independent gauge first. The propagated state must then be transformed back to the time-dependent gauge. This means that the time-evolution operator is
\begin{equation}
  \hat{U}(t_2, t_1) = \hat{U}_{F}^{\dagger}(t_2)e^{-i\hat{H}_{F}(t_2 - t_1)}\hat{U}_{F}(t_1)
\end{equation} 

Further, we enforce a periodicity constraint $\hat{U}_{F}(t) = \hat{U}_{F}(t+T)$. It is conventional to represent the operator $\hat{U}_{F}(t)$ as $e^{i\hat{K}(t)}$ and call $\hat{K}(t)$ as the kick operator.
To obtain the expressions for $\hat{H}_{F}$ and $\hat{H}_{F}$, we expand them perturbatively under the high-frequency limit.
\begin{gather}
 \hat{H}(t) = \hat{H}_{(0)} + \sum_{j=1}^{\infty}{\hat{H}_{(j)} e^{ij\omega t} + \hat{H}_{(-j)} e^{-ij\omega t}} \\
 \hat{H}_{F}(t) = \sum_{j=0}^{\infty}{\frac{1}{\omega^j}\hat{H}_{F}^{(j)}} \\
 \hat{K}(t) = \sum_{j=0}^{\infty}{\frac{1}{\omega^j}\hat{K}^{(j)}}
\end{gather}
With this ansatz, on equating terms of the same order on both sides of Eq. \eqref{chap_5:unitarytrans}, the effective Hamiltonian is determined up to a suitable order of 
$\frac{1}{\omega^n}$ \cite{goldman2014periodically}.

\section{Brillouin-Wigner perturbative expansion of Floquet Hamiltonian}
We shall exploit the periodicity of the problem and conveniently write the Schrodinger equation in terms of fourier components. Let $\{\ket{\alpha},\dots\}$ be a complete 
orthonormal set of basis states. The components of the Hamiltonian are written in this basis as $H_{\alpha\beta}=\braket{\alpha|\hat{H}|\beta}$. From results discussed in
previous sections, we know that the time-dependent Schrodinger equation has $n=rank(\hat{H})$ linearly independent solutions - one corresponding to each of the quasienergies 
in the Brillouin zone. We define the fundamental matrix of Schrodinger equation as a matrix constructed by concatenating $n$ eigenvectors as the columns of a $n \times n$ matrix.
\begin{equation*}
 \bm{\Psi}(t) = \Bigg[ \bigg[\ket{\psi_1}\bigg] \bigg[\ket{\psi_2}\bigg]\dots\bigg[\ket{\psi_n}\bigg] \Bigg]
\end{equation*}
The result of Floquet theory can be expressed in this notation as
\begin{gather}
 \bm{\Psi}(t) = \bm{\Phi}(t) e^{-i\bm{\epsilon} t} \\
 i\frac{\partial}{\partial t} \bm{\Psi}(t) = \mathbf{H}(t)\bm{\Psi}(t) \\
 \Psi_{\alpha\beta}(t) = \braket{\alpha|\psi_{\beta}(t)} \\
 \epsilon_{\alpha\beta} = \epsilon_\alpha \delta_{\alpha\beta} \\
 \Phi_{\alpha\beta}(t) = \braket{\alpha|\phi_{\beta}(t)}
\end{gather}
where $\bm{\epsilon}$ is a constant real diagonal matrix and $\bm{\Phi}(t+T)=\bm{\Phi}(t)$.

Expanding $\mathbf{H}(t)$ and $\bm{\Phi}(t)$ in fourier series
\begin{gather}
 \mathbf{H}(t) = \sum_{n}{\tilde{\mathbf{H}}_{n} e^{in\omega t}} \\
 \tilde{\mathbf{H}}_{n} = \frac{1}{T}\int_{0}^{T}{\mathbf{H}(t)e^{-in\omega t} dt} \\
 H_{\alpha\beta}(t) = \sum_{n}{\tilde{H}_{n,\alpha\beta} e^{in\omega t}} \\
 \bm{\Phi}(t) = \sum_{n}{\tilde{\bm{\Phi}}_{n} e^{in\omega t}} \\ 
 \tilde{\bm{\Phi}}_{n} = \frac{1}{T}\int_{0}^{T}{\bm{\Phi}(t)e^{-in\omega t} dt} \\
 \Phi_{\alpha\beta}(t) = \sum_{n}{\tilde{\Phi}_{n,\alpha\beta} e^{in\omega t}}
\end{gather}
Subsequently,
\begin{align}
 \Psi_{\alpha\beta}(t) &= \sum_{\gamma}{\left(\sum_{n}{\tilde{\Phi}_{n,\alpha\gamma} e^{in\omega t}}\right)\delta_{\gamma\beta}e^{-i\epsilon_\beta t}} \nonumber \\
 &= \sum_{n}{\tilde{\Phi}_{n,\alpha\beta}e^{in\omega t}e^{-i\epsilon_\beta t}}
\end{align}
Inserting these expansions into the Schrodinger equation
\begin{align*}
 i\frac{\partial}{\partial t}\Psi_{\alpha\beta}(t) &= i\frac{\partial}{\partial t}\left(\sum_{n}{\tilde{\Phi}_{n,\alpha\beta}e^{in\omega t}e^{-i\epsilon_\beta t}}\right) \\
 &= i\sum_{n}{\tilde{\Phi}_{n,\alpha\beta}e^{in\omega t}e^{-i\epsilon_\beta t} (in\omega - i\epsilon_\beta)} \\
 &= \sum_{n}{\tilde{\Phi}_{n,\alpha\beta}e^{in\omega t}e^{-i\epsilon_\beta t} (\epsilon_\beta-n\omega)} \\ 
 (H\Phi)_{\alpha\beta}(t) &= \sum_{\gamma}{\left(\sum_{r}{\tilde{H}_{r,\alpha\gamma} e^{ir\omega t}}\sum_{s}{\tilde{\Phi}_{s,\gamma\beta} e^{is\omega t}e^{-i\epsilon_\beta t}}\right)} \\
 &= \sum_{\gamma}{\sum_{r}{\sum_{s}{\tilde{H}_{r,\alpha\gamma}\tilde{\Phi}_{s,\gamma\beta}e^{i(r+s)\omega t}e^{-i\epsilon_\beta t}}}}\\
 \text{making change of variable } n = r+s \\
 &= \sum_{\gamma}{\sum_{n}{\sum_{s}{\tilde{H}_{n-s,\alpha\gamma}\tilde{\Phi}_{s,\gamma\beta}e^{in\omega t}e^{-i\epsilon_\beta t}}}} \\
 &= \sum_{n}{\left(\sum_{\gamma}{\sum_{s}{\tilde{H}_{n-s,\alpha\gamma}\tilde{\Phi}_{s,\gamma\beta}}}\right)e^{in\omega t}e^{-i\epsilon_\beta t}}
\end{align*}
Rearranging terms
\begin{equation*}
\sum_{n}\sum_{\gamma}\sum_{s}{[\tilde{H}_{n-s,\alpha\gamma} + n\omega\delta_{\alpha\gamma}\delta_{ns}]\tilde{\Phi}_{s,\gamma\beta}e^{in\omega t}e^{-i\epsilon_\beta t}}
 = \sum_{n}{\epsilon_\beta \tilde{\Phi}_{n,\alpha\beta}e^{-i\epsilon_\beta t}}
\end{equation*}
Matching coefficients of the $e^{in\omega t}$, we obtain
\begin{equation}
 \sum_{\gamma, s}{[\tilde{H}_{n-s,\alpha\gamma} + n\omega\delta_{\alpha\gamma}\delta_{ns}]\tilde{\Phi}_{s,\gamma\beta}} = \epsilon_\beta \tilde{\Phi}_{n,\alpha\beta}
\end{equation}
The above equation is an eigenvalue written the \emph{Sambe space} \cite{sambe1973steady, anisimovas2015high, shirley1965solution}, in a basis constituting states from the direct product of hilbert space $\mathcal{H}$ and 
the $\mathcal{L}_{T}$ space of square integrable periodic functions with time period $T$. 
The fourier basis forms a complete and orthonormal set of basis states for $\mathcal{L}_{T}$. We identify the basis states as $\ket{\alpha n(t)} = \ket{\alpha}e^{in\omega t}$.
Two indices or a 2-tuple $(\alpha, n)$ are/is required to identify the basis states uniquely. In this notation, we represent
\begin{equation}
 \tilde{H}^{F}_{(\alpha,n), (\gamma, s)} = \tilde{H}_{n-s,\alpha\gamma} + n\omega\delta_{\alpha\gamma}\delta_{ns}
\end{equation}
Then, the eigenvalue equation for above is written as
\begin{equation}
 \sum_{(\gamma, s)}{\tilde{H}^{F}_{(\alpha,n), (\gamma, s)}\tilde{\Phi}_{(\gamma,s), \beta}} = \epsilon_\beta\tilde{\Phi}_{(\alpha,n), \beta}
\end{equation} Note that, $\tilde{\Phi}_{\beta}$ is an eigenvector associated with eigenvalue $\epsilon_\beta$.
It is not too surprising to see that $\tilde{H}^{F}$ is actually the floquet Hamiltonian written in terms of fourier components. One major difference is that, it is
now infinite dimensional, and it allows for all quasienergies extending beyond the brillouin zone \cite{shirley1965solution, anisimovas2015high}. Another straight forward
observation is, if $\tilde{\Phi}_{(\alpha,n), \beta}$ are the components of the eigenvector corresponding to $\epsilon_\beta$, then the components of the eigenvector corresponding
to $\epsilon_\beta + m\omega$ are $\tilde{\Phi}_{(\alpha,n+m), \beta}$. This is established by the following steps
\begin{align*}
 \hat{H}_{F}\ket{\phi} &= \epsilon\ket{\phi} \\
 \hat{H}_{F}e^{im\omega t}\ket{\phi} &= (\epsilon+m\omega)e^{im\omega t}\ket{\phi}
\end{align*} 

Refer to \parencite{anisimovas2015high} for techniques on how to diagonalize or block-diagonalize the floquet Hamiltonian. Block diagonalizing the floquet Hamiltonian is
equivalent to calculating the effective Hamiltonian, as $\tilde{\mathbf{H}}_{n-s} = \frac{1}{T}\int_{0}^{T}{\hat{H}(t) e^{-i(n-s)\omega t}}$ is non-zero only when
$n=s$, because, $\hat{H}$ under the unitary transformation, is static in time.

Next step in our discourse is to apply Brillouin-Wigner perturation theory (See Chapter \ref{ch:bw}) to diagonalize the floquet Hamiltonian. This is a very recent technique developed by Mikami et al. in 
\parencite{mikami2016brillouin}. We begin with the floquet eigenvalue equation, written in matrix form \footnote{In this analysis, the definition of $\ket{m(t)}$ is 
inverted to $e^{-im\omega t}$ to match the notation used by the original authors \cite{mikami2016brillouin}.}
\begin{gather}
(\mathcal{H} - \mathcal{M}\omega)\ket{\phi_\alpha} = \epsilon_\alpha\ket{\phi_\alpha} \\
\mathcal{H}_{mn} = \frac{1}{T}\int_{0}^{T}{\hat{H}(t)e^{i(m-n)\omega t}dt} \\
\mathcal{M}_{mn} = m\delta_{mn}
\end{gather}
Partitioning is done with the choice of $\mathcal{P}_{mn} = \delta_{mn}\delta_{m0}$, which essentially maps the eigenfunctions into a space devoid of the micromotion information.
As a result, $\mathcal{Q}_{mn} = 1 - \mathcal{P}_{mn} = \delta_{mn}(1-\delta_{m0})$, $(\mathcal{P}\mathcal{M})_{mn} = k\delta_{mk}\delta_{m0}\delta_{kn}=0$ and
$(\mathcal{Q}\mathcal{M})_{mn} = k\delta_{mk}(1-\delta_{m0})\delta_{kn} = m\delta_{mk}\delta_{kn}(1-\delta_{k0}) = (\mathcal{M}\mathcal{Q})_{mn}$.
The Brillouin-wigner effective Hamiltonian is defined as
\begin{equation}
 H_{eff} = \mathcal{P}(\mathcal{H} - \mathcal{M}\omega)\Omega\mathcal{P}
\end{equation} We determine the $\Omega$ wave operator as follows
\begin{align}
 \mathcal{Q}(\mathcal{H} - \mathcal{M}\omega)\ket{\phi_\alpha} &= \epsilon_\alpha\mathcal{Q}\ket{\phi_\alpha} \nonumber \\
 \mathcal{Q}\mathcal{H}\ket{\phi_\alpha} &= \mathcal{Q}\mathcal{M}\omega\ket{\phi_\alpha} + \epsilon_\alpha\mathcal{Q}\ket{\phi_\alpha} \nonumber \\
 \mathcal{Q}\mathcal{H}\ket{\phi_\alpha} &= \mathcal{M}\omega\mathcal{Q}\ket{\phi_\alpha} + \epsilon_\alpha\mathcal{Q}\ket{\phi_\alpha} \nonumber \\
 (\epsilon_\alpha + \mathcal{M}\omega)\mathcal{Q}\ket{\phi_\alpha} &= \mathcal{Q}\mathcal{H}\ket{\phi_\alpha} \nonumber \\
 \mathcal{Q}\ket{\phi_\alpha} &= \frac{\mathcal{Q}}{\epsilon_\alpha + \mathcal{M}\omega}\mathcal{H}\ket{\phi_\alpha} \\
\end{align}
\begin{align}
 \ket{\phi_\alpha} &= \mathcal{P}\ket{\phi_\alpha} + \mathcal{Q}\ket{\phi_\alpha} \nonumber \\
 &= \mathcal{P}\ket{\phi_\alpha} + \frac{\mathcal{Q}}{\epsilon_\alpha + \mathcal{M}\omega}\mathcal{H}\ket{\phi_\alpha} \nonumber \\
 \mathcal{P}\ket{\phi_\alpha} &= \ket{\phi_\alpha} - \frac{\mathcal{Q}}{\epsilon_\alpha + \mathcal{M}\omega}\mathcal{H}\ket{\phi_\alpha}  \nonumber \\
 \ket{\phi_\alpha} &= \left(1 - \frac{\mathcal{Q}}{\epsilon_\alpha + \mathcal{M}\omega}\mathcal{H}\right)^{-1}\mathcal{P}\ket{\phi_\alpha}
\end{align}
We recognize the wave operator from the above expression as
\begin{equation}
 \Omega(\epsilon) = \left(1 - \frac{\mathcal{Q}}{\epsilon + \mathcal{M}\omega}\mathcal{H}\right)^{-1}
\end{equation} and consequently the effective Hamiltonian as
\begin{align}
 H_{eff}(\epsilon) &= \mathcal{P}(\mathcal{H} - \mathcal{M}\omega)\left(1 - \frac{\mathcal{Q}}{\epsilon + \mathcal{M}\omega}\mathcal{H}\right)^{-1}\mathcal{P} \nonumber\\
 &= \mathcal{P}\mathcal{H}\left(1 - \frac{\mathcal{Q}}{\epsilon + \mathcal{M}\omega}\mathcal{H}\right)^{-1}\mathcal{P} - \omega\mathcal{P}\mathcal{M}\left(1 - \frac{\mathcal{Q}}{\epsilon + \mathcal{M}\omega}\mathcal{H}\right)^{-1}\mathcal{P} \nonumber\\
 &= \mathcal{P}\mathcal{H}\left(1 - \frac{\mathcal{Q}}{\epsilon + \mathcal{M}\omega}\mathcal{H}\right)^{-1}\mathcal{P}
\end{align} 

We look to extend this theory by defining an $\epsilon$-independent effective Hamiltonian.
\begin{align*}
 \Omega(\epsilon) = \Omega(\epsilon)\mathcal{P} &= \left(1 - \frac{\mathcal{Q}}{\epsilon + \mathcal{M}\omega}\mathcal{H}\right)^{-1} \mathcal{P} \\
 \left(1 - \frac{\mathcal{Q}}{\epsilon + \mathcal{M}\omega}\mathcal{H}\right)\Omega(\epsilon) &= \mathcal{P} \\
 \Omega(\epsilon) - \frac{\mathcal{Q}}{\epsilon + \mathcal{M}\omega}\mathcal{H}\Omega(\epsilon) &= \mathcal{P} \\
 (\epsilon + \mathcal{M}\omega)\Omega(\epsilon) - \mathcal{Q}\mathcal{H}\Omega(\epsilon) &= \epsilon\mathcal{P} + \mathcal{M}\omega\mathcal{P} \\
 \Omega(\epsilon)\epsilon - \mathcal{P}\epsilon + \mathcal{M}\omega\Omega(\epsilon) - \mathcal{Q}\mathcal{H}\Omega(\epsilon) &= \mathcal{M}\omega\mathcal{P} \\
 \Omega(\epsilon)\epsilon - \mathcal{P}\Omega(\epsilon)\epsilon + \mathcal{M}\omega\Omega(\epsilon) - \mathcal{Q}\mathcal{H}\Omega(\epsilon) &= \mathcal{M}\omega\mathcal{P} \\
 (1-\mathcal{P})\Omega(\epsilon)\epsilon + \mathcal{M}\omega\Omega(\epsilon) - \mathcal{Q}\mathcal{H}\Omega(\epsilon) &= \mathcal{M}\omega\mathcal{P} \\
 \mathcal{Q}\Omega(\epsilon)\epsilon + \mathcal{M}\omega\Omega(\epsilon) - \mathcal{Q}\mathcal{H}\Omega(\epsilon) &= \mathcal{M}\omega\mathcal{P} \\
 \mathcal{M}\omega\mathcal{P} + \mathcal{Q}\mathcal{H}\Omega(\epsilon) - \mathcal{Q}\Omega(\epsilon)\epsilon &= \mathcal{M}\omega\Omega(\epsilon) \\ 
\end{align*}
\begin{equation}
 \Omega(\epsilon) = \mathcal{P} + \frac{\mathcal{Q}}{\mathcal{M}\omega}\mathcal{H}\Omega(\epsilon) - \frac{\mathcal{Q}}{\mathcal{M}\omega}\Omega(\epsilon)\epsilon
\end{equation} We can replace the $\epsilon$ in the above equation with $H_{eff}$.
\begin{equation*}
 \Omega(\epsilon) = \mathcal{P} + \frac{\mathcal{Q}}{\mathcal{M}\omega}\mathcal{H}\Omega(\epsilon) - \frac{\mathcal{Q}}{\mathcal{M}\omega}\Omega(\epsilon)\mathcal{P}\mathcal{H}\Omega(\epsilon)\mathcal{P}
\end{equation*} We omit the $\mathcal{P}$ in the last term, as it $\Omega$ is anyway acted on $\mathcal{P}\ket{\phi_\alpha}$.

$\epsilon$ independent wave operator is defined by the recursive relation
\begin{equation}
 \label{chap_5:omegabw}\Omega(\epsilon) = \mathcal{P} + \frac{\mathcal{Q}}{\mathcal{M}\omega}\mathcal{H}\Omega(\epsilon) - \frac{\mathcal{Q}}{\mathcal{M}\omega}\Omega(\epsilon)\mathcal{P}\mathcal{H}\Omega(\epsilon)
\end{equation} and the $\epsilon$ independent effective Hamiltonian is obtained from the solution $\Omega_{BW}$, of the preceding equation.
\begin{equation}
 H_{BW} = \mathcal{P}\mathcal{H}\Omega_{BW}\mathcal{P}
\end{equation}

$\Omega_{BW}$ is obtained by substitution of the $1/\omega$ series, 
\begin{equation*}
 \Omega_{BW} = \sum_{N=0}^{\infty}{\Omega_{BW}^{(N)}}
\end{equation*} where $\Omega_{BW}^{(N)}$ corresponds to order $1/\omega^N$ coefficient in the iterative solution to $\Omega_{BW}$. Similarly, the effective Hamiltonian
is also expanded in a series \begin{equation*}
 H_{BW} = \sum_{N=0}^{\infty}{H_{BW}^{(N)}}
\end{equation*} and $H_{BW}^{(N)} = \mathcal{P}\mathcal{H}\Omega_{BW}^{(N)}\mathcal{P}$. Expansion up to 4th order can be found in \parencite{mikami2016brillouin}.

Under the limit $\omega \rightarrow \infty$, the operators tend to their $0th$ order terms, $\Omega_{BW} \rightarrow \mathcal{P}$ and $H_{BW} \rightarrow H_{0,0}$. This guarantees that the eigenvalues of 
$H_{BW}$ are in the first brillouin zone, as the contributions from higher order terms in the series is very small to transport the eigenvalues to the next brillouin zone.

The wave operator obtained from the above discussion was written in terms of the fourier components. Using fourier series, we can go back to time domain and express our 
wavefunctions as a function of time. This operator is defined as
\begin{equation}
 \Xi(t) = \sum_{n\in\mathbb{Z}}{e^{-in\omega t}[\Omega_{BW}]_{n,0}}
\end{equation}
and
\begin{equation}
 \ket{\psi_{\alpha}(t)} = e^{-i\epsilon_\alpha t}\Xi(t)\ket{\phi_\alpha^0}
\end{equation}

A nuanced discussion of this theory and its connections to other approximation techniques based on the floquet theory can be found in \parencite{mikami2016brillouin}.
% \begin{gather}
%  \Omega_{BW}^{(0)} = \mathcal{P} \\
%  \Omega_{BW}^{(1)} = \frac{\mathcal{Q}}{\mathcal{M}\omega}\mathcal{H}\mathcal{P} \\
% \end{gather}
