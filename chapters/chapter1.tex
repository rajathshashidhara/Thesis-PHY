% Author	Rajath Shashidhara
% Email		rajath.shashidhara@gmail.com
%
% This work is licensed under the Creative Commons Attribution 4.0 International License. 
% To view a copy of this license, visit http://creativecommons.org/licenses/by/4.0/.

\chapter{Introduction}
% \section{section}
% \subsection{subsection}
The major focus of this thesis is the study of electron motion in 2D lattice in a perpendicular constant magnetic field. This problem is interesting for various reasons to
theoretical physicists, experimental/applied physicists and mathematicians alike. Superficially, it seems like a simple problem, the kind that is used as an example problem in a textbook. In reality,
this problem displays a picturesque spectrum having profound number theoretic properties. Historically, this problem is closely associated with Quantum Hall effect and it is the first system
for which the topological connection to Hall conductivity was established.

Quantum Hall effect - a mysterious phenomenon of quantized Hall conductivity - was hotly debated by condensed matter theorists in the 1970s. This quantization is not a direct
result of application of Quantum mechanics on the system. This intriguing effect was convincingly explained by Thouless et al., when they unveiled the connection of topology to
Hall conductance in 2-dimensional periodic lattice. This sparked the study of topological phases of matter and the quest for topological phase transitions. Concurrently, Berry
discovered a purely geometric contribution to phase factor accumulated by the state in adiabatic evolution. This phase was physically observable and independent of the gauge choice.
The shroud of mystery surrounding the Berry phase was lifted when Simon demonstrated that Berry phase is in fact connected to the discovery of Thouless. 

The objective of this thesis is to periodically drive AAH models and analyze the effect of such a perturbation. Periodic driving is studied using the Floquet formalism.
Using Floquet theory, a class of perturbative techniques have been developed to tackle periodic driving. The study of cold atoms in optical lattices are used as experimental
setups to create artificial gauge fields to simulate the AAH model. These techniques crucially rely on the Floquet theory. We shall study the driven AAH models using Floquet
theory and probe the system for topological transitions.

In Chapter \ref{ch:gp}, Berry phase is introduced. A brief introduction to Quantum Hall effect and its connection to Berry phase is presented in Chapter \ref{ch:qhe}. Chapters
\ref{ch:bw} and \ref{ch:fl} comprehensively cover the Floquet theory and its application to driven quantum systems. Finally, in Chapter \ref{ch:aah} covers the pure AAH model comprehensively.
In the later chapters, the driven AAH models are introduced and the results are stated and analyzed.