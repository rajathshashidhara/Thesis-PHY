% Author	Rajath Shashidhara
% Email		rajath.shashidhara@gmail.com
%
% This work is licensed under the Creative Commons Attribution 4.0 International License. 
% To view a copy of this license, visit http://creativecommons.org/licenses/by/4.0/.

\chapter{Brillouin-Wigner Perturbation Theory} \label{ch:bw}
When encountered with an analytically intractable quantum mechanical problem, perturbation techniques 
may be used to obtain approximate solutions. Formally, the problem can be stated as follows : 
Given a hamiltonian $\hat{H} = \hat{H}_{0} + \hat{V}$ where $\hat{H}_{0}$ is exactly solvable and 
$\hat{V}$ is the perturbation term, and the eigendecomposition of $\hat{H}_{0}\quad$ \textemdash 
\begin{gather*}
  \hat{H}_{0}\ket{n} = \epsilon_{n}\ket{n} \\
  \braket{m|n} = \delta_{mn} \\
  \sum_{m} {\ket{m} \bra{m}} = \mathrm{1}
\end{gather*}
Using Brillouin-Wigner (BW) perturbation theory, we express $\{\ket{\psi_n},\dots\}$ and $\{E_n\,,\dots\}$, such that $\hat{H}\ket{\psi_n} = E_n\ket{\psi_n}$, in terms of
$\hat{V}$ and $\{\ket{n},\dots\}$.

\section{Introduction}
To obtain the BW perturbative expansion, we begin with the eigenvalue equation. 
\begin{equation*}
 \hat{H}\ket{\psi_n} = (\hat{H}_{0} + \hat{V})\ket{\psi_n} = E_n\ket{\psi_n}
\end{equation*}
The wavefunctions $\ket{\psi_n}$ are normalized as $\braket{n|\psi_n} = 1$, as discussed in \cite{sakurai2011modern}. On 
contracting with $\bra{n}$,
\begin{gather}
 \braket{n|(\hat{H}_{0} + \hat{V})| \psi_n} = E_n\braket{n|\psi_n} \nonumber \\
 \epsilon_n \braket{n|\psi_n} + \braket{n|\hat{V}|\psi_n} = E_n\braket{n|\psi_n} \nonumber \\
 \label{chap_4:eneryeq} E_n = \epsilon_n + \braket{n|\hat{V}|\psi_n}
\end{gather}
Rewriting the eigenvalue equation as
\begin{align*}
 (E_n - \hat{H}_{0})\ket{\psi_n} &= \hat{V}\ket{\psi_n} \\
 &= \mathrm{1}\; \hat{V}\ket{\psi_n} \\
 &= \sum_{m}{\ket{m}\bra{m}}\; \hat{V}\ket{\psi_n} \\
 &= \ket{n}\braket{n|\hat{V}|\psi_n} + (\mathrm{1} - \ket{n}\bra{n})\hat{V}\ket{\psi_n} \\
 \text{Using Eq. \ref{chap_4:eneryeq},} \\
 &= (E_n - \hat{H}_{0})\ket{n} + (\mathrm{1} - \ket{n}\bra{n})\hat{V}\ket{\psi_n} \\
 (E_n - \hat{H}_{0}) (\ket{\psi_n} - \ket{n}) &= (\mathrm{1} - \ket{n}\bra{n})\hat{V}\ket{\psi_n} \\
 \ket{\psi_n} &= \ket{n} + (E_n - \hat{H}_{0})^{-1} (\mathrm{1} - \ket{n}\bra{n})\hat{V}\ket{\psi_n}
\end{align*}
Define resolvent operator as $\hat{R}_{n} = (E_n - \hat{H}_{0})^{-1} = \sum_{n}{\ket{n}(E_n - \epsilon_n)^{-1} \bra{n}}$,
\begin{equation}
  \label{chap_4:bweqn}\ket{\psi_n} = \ket{n} +  \hat{R}_{n}(\mathrm{1} - \ket{n}\bra{n})\hat{V}\ket{\psi_n}
\end{equation}
The above equation is the main result of BW perturbation theory. Solving Eq. \eqref{chap_4:bweqn} self-consistently with 
Eq. \eqref{chap_4:eneryeq}, solutions to the eigenvalue equation are obtained. Using the equation, we obtain an iterative solution to the schordinger equation. 
In each iteration, we obtain a new estimate of $\ket{\psi_n}$ by substituting the old estimate of $\ket{\psi_n}$ on the righthand side of Eq. \eqref{chap_4:bweqn}
\begin{gather}
  \label{chap_4:itereq} \ket{\psi_{n}^{(j)}} = \ket{n} + \; \hat{R}(E_{n})\,(\hat{\mathbb{I}} - \ket{n}\bra{n})\hat{V}\ket{\psi_{n}^{(j-1)}} \\
  \ket{\psi_{n}^{(0)}} = \ket{n} \nonumber \\
  \lim_{j\rightarrow \infty} \ket{\psi_{n}^{(j)}} = \ket{\psi_n} \nonumber
\end{gather} No approximation has been used until this point, and exact solution can be obtained if iterated infinitely. 
In practice, approximate solution is obtained by truncating the iteration.

\section{BW theory as an Operator series}
Further, Eq. \eqref{chap_4:bweqn} can be simplified by expanding the recurrence relation
\begin{align}
 \ket{\psi_n} &= \ket{n} + \hat{R}_{n}\hat{Q}_{n}\hat{V}\ket{\psi_n} \nonumber\\
 \text{where } \hat{Q}_{n} = \mathrm{1} - \ket{n}\bra{n}, \nonumber\\
 &= \ket{n} + \hat{R}_{n}\hat{Q}_{n}\hat{V}\ket{n} + \hat{R}_{n}\hat{Q}_{n}\hat{V}\hat{R}_{n}\hat{Q}_{n}\hat{V}\ket{n} + \dots \nonumber\\
 &= \sum_{k=0}^{\infty}{\{\hat{R}_{n}\hat{Q}_{n}\hat{V}\}^k \ket{n}} \nonumber\\ 
 &= (\mathrm{1} - \{\hat{R}_{n}\hat{Q}_{n}\hat{V}\})^{-1} \ket{n}
\end{align}
When higher order term contributions are diminishingly small, truncating the series produces approximate solutions 
to the problem. However, in certain specialized problem settings, operator inverse in the above equation can be analytically obtained \cite{silvert1972comparison}.

\section{Comparison with Rayleigh-Schrodinger theory}
Rayleigh-Schrodinger (RS) perturbation theory is a more popular perturbation theory and is used widely in practice. But, RS theory is not without its shortcomings. 
Unlike the RS theory, BW theory does not require separate treatment in the case degenerate spectrum. A thorough comparative study of RS and BW theory can be found in \parencite{silvert1972comparison}. 
In fact, RS theory can be obtained as an approximation to BW theory by power series expansion of the resolvent operator \cite{wilson2009brillouin}. A major problem with
the BW theory is that Eq. \eqref{chap_4:bweqn} is an implicit equation in $E_n$ and it must be solved in conjunction with Eq. \eqref{chap_4:eneryeq}. Solving this system of
equations is usually tricky as compared to the simple perturbation terms obtained from RS theory.

\section{Effective Hamiltonian}
Recent efforts to extend the theory to many-body systems has led to systematization of BW theory in 
terms of model space and effective hamiltonian formalism. This new representation requires the
introduction of a model space with respect to a set of reference states. Any complete set of orthonormal states of hilbert space 
is chosen as a set of reference states. Usually, eigenstates of the unperturbed hamiltonian $\hat{H}_{0}$ is chosen as a set of 
reference states. The hilbert space is partitioned into model space and orthogonal space, by choosing one state from the set of
reference states as model state \footnote{In this treatment, a single reference state is chosen as the model function. See 
\cite{wilson2009brillouin} for multi-reference partitioning.}.

Let $\mathrm{R}\equiv\{\ket{\phi_n}\dots\}$ is the set of reference states, $\ket{\phi_0} \in \mathrm{R}$ is the model state, 
then $P = \ket{\phi_0}\bra{\phi_0}$ is the corresponding projection operator of the model space and $Q = \mathrm{1} - P$ is the
projection operator corresponding to orthogonal space. A state $\ket{\psi}$ in the hilbert space can be projected onto the model 
space using operator $P$, $\ket{\phi} = P\ket{\psi}$ and a wavefunction $\ket{\phi}$ in model space can be reconstructed in hilbert
space using the wave operator $\Omega$ as $\ket{\psi} = \Omega\ket{\phi}$.

Some useful relationships between operators $P$, $Q$ and $\Omega$ are 
\begin{enumerate}
 \item $P + Q = \mathrm{1}$ (by definition)
 \item $P^2 = P$ and $Q^2 = Q$ (property of projection operators)
 \item $PQ = QP = 0$ (using Property 1)
 \item $\Omega^2\ket{\phi} = \Omega\ket{\phi}$ (by definition)
 \item $\Omega P \ket{\phi} = \Omega \ket{\phi}$ and $P\Omega \ket{\psi} = P \ket{\psi}$ (by definition)
\end{enumerate}

Provided the eigenvalue equation
$\hat{H}\ket{\psi} = E\ket{\psi}$, then $\ket{\phi} = P\ket{\psi}$ satisfies the equation 
$\hat{H}_{eff}\ket{\phi} = E\ket{\phi}$, where 
\begin{equation}
  \hat{H}_{eff} = P\hat{H}\Omega P
\end{equation}
\begin{align*}
 \hat{H}_{eff}\ket{\phi} &= P\hat{H}\Omega P \ket{\phi} \\
 &= P\hat{H}\Omega P P\ket{\psi} \\
 &= P\hat{H}\ket{\psi} \\
 &= E\,P\ket{\psi} \\
 &= E\ket{\phi}
\end{align*}

Therefore, the eigenvalues of the effective hamiltonian are equal to the eigenvalues of the original hamiltonian and the 
eigenfunctions of the original hamiltonian can be obtained by application of the wave operator $\Omega$ on the eigenfunctions of
the effective hamiltonian $\ket{\psi} = \Omega\ket{\phi}$.

How do we obtain the wave operator $\Omega$ ? Operating $Q$ on the eigenvalue equation,
\begin{align*}
 Q\hat{H}\ket{\psi} &= E\,Q\ket{\psi} \\
 Q\ket{\psi} &= \frac{Q\hat{H}}{E}\ket{\psi}
\end{align*}
Therefore,
\begin{align}
  \ket{\psi} &= (P + Q)\,\ket{\psi} \nonumber \\
  &= P\ket{\psi} + Q\ket{\psi} \nonumber \\
  \label{chap_4:effform} &= P\ket{\psi} + \frac{Q\hat{H}}{E}\ket{\psi} \\
  P\ket{\psi} &= \left(1-\frac{Q\hat{H}}{E}\right)\ket{\psi} \nonumber \\
  \label{chap_4:waveop} \ket{\psi} &= \left(1-\frac{Q\hat{H}}{E}\right)^{-1} P^2 \ket{\psi}  
\end{align} Eq. \eqref{chap_4:effform} has the familiar iterative form as Eq. \eqref{chap_4:bweqn} with the choice of 
$P = \ket{n}\bra{n}$. From Eq. \eqref{chap_4:waveop}, 
\begin{equation}
 \Omega = \left(1-\frac{Q\hat{H}}{E}\right)^{-1}P
\end{equation}
The inverse operation in the wave operator can be expanded to obtain the perturbative expansion.

Usually, the effective hamiltonian is parameterized by energy. To obtain the solutions, we must diagonalize the effective hamiltonian by treating $E$ as a free parameter to
obtain the eigenvalue expressions $E_{i} (i=1\dots \dim(H))$, and solve the equations $E = E_{i}$ to obtain the numerical values of energy eigenvalues.

Above discussion is an abridged version of the theory of BW perturbation presented in \parencite{wilson2009brillouin}.