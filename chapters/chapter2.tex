% Author	Rajath Shashidhara
% Email		rajath.shashidhara@gmail.com
%
% This work is licensed under the Creative Commons Attribution 4.0 International License. 
% To view a copy of this license, visit http://creativecommons.org/licenses/by/4.0/.

\chapter{Geometric Phase}\label{ch:gp}
Physical measurements or observations are extracted from quantum mechanical systems in the form of expectation values of hermitian operators $\braket{\psi|\hat{A}|\psi}$.
Eigenvectors of the hamiltonian are only specified up to a phase factor. Expectation values and thereby, the physical observations are unaltered by a transformation of the form 
$\psi \rightarrow e^{i \phi} \psi$. Gauge-dependent quantities are considered unphysical and the phase of the eigenvectors is usually thrown away from analysis. Berry
demonstrated the existence of a gauge-invariant phase called the geometric phase - a physically observable quantity.
% This guage choice can be eliminated by formulating quantum mechanics in terms of guage invariant representations such as density 
% matrices $\psi\psi^{\dagger}$. Expectation value of an operator $\hat{A}$ is recast as $\braket{\hat{A}} = tr(\psi\psi^{\dagger}\hat{A})$.

\section{Definition}
Prof. M V Berry studied the adiabatic evolution of a quantum system described by the hamiltonian $\hat{H}(\mathbf{R})$, parameterized by external factors $\mathbf{R}$ -- such as
magnetic field, electric field, etc \cite{berry1984quantal}. The hamiltonian is varied by slowly changing $\mathbf{R}$ from time $t=0$ to $t=T$ such that $\mathbf{R}(0)=\mathbf{R}(T)$. This evolution is
a closed loop in the parameter space. Apart from the adiabatic assumption, we also assume that the energy spectrum is discrete and there are no level crossings or degeneracies in the 
energy spectrum. Using the guage freedom, we enforce single-valuedness on the eigenvectors of the hamiltonian in the parameter space. 
Under the assumptions stated above, eigenvectors of the hamiltonian can be uniquely identified as $\ket{n(\mathbf{R})}$ at each point in parameter space without any ambiguity.
The schrodinger equation is stated in this notation as 
\begin{gather}
 \hat{H}(\mathbf{R}(t))\ket{n(\mathbf{R}(t))} = E(\mathbf{R}(t))\ket{n(\mathbf{R}(t))} \\
 \label{chap_2:timedepschroeq} i\hbar\frac{\partial}{\partial t}\ket{\psi(t)} = \hat{H}(t)\ket{\psi(t)}
\end{gather} In the case of static time-independent hamiltonian, any state expanded as a linear combination of stationary states is written as
$\ket{\psi(t)} = \sum{c_n e^{-\frac{i E_{n}(t)}{\hbar}} \ket{n}}$. Taking cue from this, the expansion ansatz
\begin{equation}
\ket{\psi(t)} = \sum{c_{n}(t) e^{-\frac{i}{\hbar}\int_{0}^{t}{E_{n}(t) dt}} \ket{n(t)}}
\end{equation} is a valid generalization for a system with a time-dependent hamiltonian. When the ansatz is substituted into Eq. \eqref{chap_2:timedepschroeq}, we obtain
\begin{equation}
  \label{chap_2:coeffslinercomb}\dot{c}_{m}(t) = -c_{m}(t) \Braket{m(t) | \frac{\partial}{\partial t} m(t)} - \sum_{n}{c_{n}(t) e^{i (\theta_n - \theta_m)} \frac{\braket{m(t)|\widehat{\dot{H}}|n(t)}}{E_n - E_m}}
\end{equation} where $\theta_n = \frac{-1}{\hbar}\int_{0}^{t}{E_n(t')\,dt'}$. In the adiabatic limit, the transition probability between states tends to zero.
We ignore the second term in the general solution above to obtain the result
\begin{equation}
 c_{n}(t) = e^{i \gamma_{n}(t)}\,c_{n}(0)
\end{equation} If we started in an eigenstate of the hamiltonian $\ket{\psi(0)} = \ket{n(\mathbf{R}(0)}$, then the state evolves as
\begin{equation}
 \ket{\psi(t)} = e^{i \gamma_{n}(t)} e^{i \theta_{n}(\mathbf{R}(t))} \ket{n(\mathbf{R}(t))}
\end{equation} remaining in the $n$th eigenstate all along, only picking up phase factors. In the case of cyclic evolution, we write
\begin{equation}
 \ket{\psi(T)} = e^{i \gamma_{n}(C)} e^{i \theta_{n}(T)} \ket{\psi(0)}
\end{equation}
From Eq. \eqref{chap_2:coeffslinercomb}, we note that
\begin{equation}
  \gamma_{n}(C) = i\,\int_{0}^{T}{\braket{n(t) | \frac{\partial}{\partial t} n(t)}\, dt} = i\, \oint_{c}{\braket{n(\mathbf{R})| \bm{\nabla_{R}} | n(\mathbf{R})} \cdot \, d\mathbf{R}}
\end{equation} $\gamma_{n}(C)$ is a real number and therefore $e^{i \gamma_{n}(C)}$ is a pure phase term. This simple fact can be demonstrated by taking the time-derivative 
of $\braket{n(t)|n(t)} = 1$.
\begin{gather*}
 \braket{n(t)|n(t)} = 1 \\
 \left[\frac{\partial}{\partial t}\bra{n(t)}\right]\ket{n(t)} + \bra{n(t)}\left[ \frac{\partial}{\partial t}\ket{n(t)}\right] = 0 \\
 \bra{n(t)}\left[\frac{\partial}{\partial t}\ket{n(t)}\right] = -\left(\bra{n(t)}\left[\frac{\partial}{\partial t}\ket{n(t)}\right]\right)^{*}
\end{gather*}
$\theta_{n}(T)$ is the familiar dynamical phase and $\gamma_{n}(C)$ is known as geometric phase or simply berry phase. While the dynamical phase is a time-dependent quantity,
the geometric phase is a line integral, that only depends on path traversed in the parameter space. Another important observation is the fact that $\gamma_{n}$ is a 
non-integrable quantity as it is not single valued - $\gamma_{n}(0) \neq \gamma_{n}(T)$ in the parameter space. Therefore, the geometric phase cannot be expressed as a field over
the parameter space. By application of stokes theorem, the line integral can be convereted into a surface integral as shown below
\begin{align}
 \gamma_{n}(C) &= i \oint_{C}{\braket{n(\mathbf{R})| \bm{\nabla_{R}} | n(\mathbf{R})} \cdot \, d\mathbf{R}} \nonumber \\
 &= i \int_{S}{\bm{\nabla_{R}} \wedge \braket{n(\mathbf{R})| \bm{\nabla_{R}} | n(\mathbf{R})} \cdot \, d\mathbf{S}} \\
 &= i \int_{S}{(\bm{\nabla_{R}}\bra{n(\mathbf{R})}) \wedge \bm{\nabla_{R}}\ket{n(\mathbf{R})} \cdot \, d\mathbf{S}} \nonumber \\
 \label{chap_2:gaugeindep} &= i \bigintsss_{S}{\,\sum_{m \neq n}{\frac{\braket{n(\mathbf{R})|\bm{\nabla_{R}}\hat{H}(\mathbf{R})|m(\mathbf{R})} \wedge \braket{m(\mathbf{R})|\bm{\nabla_{R}}\hat{H}(\mathbf{R})|n(\mathbf{R})}}{(E_{n}(\mathbf{R})- E_{m}(\mathbf{R}))^2}} \cdot \, d\mathbf{S}}
\end{align}
Eq. \eqref{chap_2:gaugeindep} has a remarkable property of being gauge independent. Any transformation of the form 
$\ket{n(\mathbf{R})} \rightarrow e^{i\delta(\mathbf{R})} \ket{n(\mathbf{R})}$, leaves Eq. \eqref{chap_2:gaugeindep} unchanged. 

Before we proceed, it is useful to define two important quantities called the berry connection and the berry curvature. \emph{Berry connection} is defined as
\begin{equation}
 \mathbf{A}_{n}(\mathbf{R}) = i\braket{n(\mathbf{R})| \bm{\nabla_{R}} | n(\mathbf{R})}
\end{equation}
and the \emph{Berry curvature} is given by
\begin{equation}
 \mathbf{B}_{n}(\mathbf{R}) = i \;\bm{\nabla_{R}} \wedge \braket{n(\mathbf{R})| \bm{\nabla_{R}} | n(\mathbf{R})} = \bm{\nabla_{R}} \wedge \mathbf{A}_{n}(\mathbf{R})
\end{equation} Berry phase is conveniently expressed in terms of these quantities as
\begin{equation}
 \gamma_{n}(C)= \oint_{C}{\mathbf{A}_{n}(\mathbf{R})\cdot \, d\mathbf{R}} = \int_{S}{\mathbf{B}_{n}(\mathbf{R})\cdot \, d\mathbf{S}}
\end{equation}
If we transform the states as $\ket{n(\mathbf{R})} \rightarrow e^{i\delta(\mathbf{R})} \ket{n(\mathbf{R})}$, then the berry connection transforms as
$\mathbf{A}_{n}(\mathbf{R}) \rightarrow \mathbf{A}_{n}(\mathbf{R}) - \bm{\nabla_{R}} \delta(\mathbf{R})$, but the berry curvature remains unchanged. This shows an uncanny
resemblance between magnetic vector potential and the berry connection and consequently the magnetic field strength and the berry curvature. However, it must be kept in mind that
the berry connection and the berry curvature are defined over the parameter space (could be more than 4-dimensional) and not the real space.

Geometric phase was overlooked as physically irrelevant as the expectation values are independent of the phase of eigenfunctions. 
As opposed to the then prevalent notion, Prof. Berry, in his seminal paper, demonstrated that the geometric phase is in fact, physically observable \cite{berry1984quantal,wilczek1989geometric}.
Several assumptions imposed by Prof. Berry in defining the geometric phase can be relaxed and definition of geometric phase has been generalized to nonadiabatic, noncyclic, 
nonunitary and nonabelian situations \cite{aharonov1987phase, samuel1988general, wilczek1984appearance, mukunda1993quantum}.

\section{Bargmann invariants}
A remarkable feature of the berry phase is that it is a geometric quantity, i.e., it is a property arising by virtue of the geometry of the 
underlying hilbert space. In \parencite{mukunda1993quantum}, it is demonstrated that the berry phase is both gauge and reparameterization 
invariant (thereby a geometric property) using purely kinematic arguments.

To each point in the parameter space, we associate a 1-dimensional complex vector space (spanned by all gauge choices of the eigenstate at that point). We call the parameter space a vector bundle for this reason.
When we try to compare vectors from vector spaces associated with two different points in the parameter space, we need to use the Berry connection (akin to Levi-Cevita connection/
Christoffel symbols in General relativity). Just as the co-variant derivative of a vector on a manifold has contributions purely arising from the underlying 
geometry, the geometric phase is acquired by a state evolving in the hilbert space by virtue of its geometry \cite{simon1983holonomy}. 
Precisely, berry phase is the holonomy in a hermitian line bundle in the language of differential geometry.

In this section, our intention is to express the berry phase in terms of gauge invariant quantities called Bargmann invariants. 
Bargmann invariant of order $j$ of a series of $j$ normalized states $\ket{\psi_j}$ such that $\braket{\psi_j|\psi_{j+1}} \neq 0$ is defined as 
\begin{equation}
 \Delta^{(j)}(\psi_1, \dots, \psi_j) = \braket{\psi_1 | \psi_2}\braket{\psi_2|\psi_3}\dots\braket{\psi_{j-1}|\psi_j}\braket{\psi_j|\psi_1}
\end{equation} It is apparent from the definition that $\Delta^{(j)}$ is invariant under a $U(1) \times U(1) \dots j\;times$ transformation.
Consider the space of eigenfunctions parameterized (under the assumptions as 
described in the previous section) by $\mathbf{R}$, the inner product 
\begin{align}
e^{i\Delta\gamma} &= \frac{\braket{n(\mathbf{R})|n(\mathbf{R} +\delta\mathbf{R})}}{|\braket{n(\mathbf{R})|n(\mathbf{R} +\delta\mathbf{R})}|} \\
 		  &\approx \frac{\bra{n(\mathbf{R})}\;\lbrack \ket{n(\mathbf{R})} + \bm{\nabla_R}\ket{n(\mathbf{R})}\cdot \delta\mathbf{R}\, + \dots \rbrack}{|\braket{n(\mathbf{R})|n(\mathbf{R} +\delta\mathbf{R})}|} \nonumber \\
1 + i\Delta\gamma + \dots &\approx \frac{1 + \braket{n(\mathbf{R})|\bm{\nabla_R}|n(\mathbf{R})}\cdot \delta\mathbf{R} + \dots}{|\braket{n(\mathbf{R})|n(\mathbf{R} +\delta\mathbf{R})}|} \nonumber \\
\Delta\gamma	   &\approx  -i\braket{n(\mathbf{R})|\bm{\nabla_R}|n(\mathbf{R})}\cdot \delta\mathbf{R}
\end{align}
We have established that 
\begin{equation*}
 \arg(\braket{n(\mathbf{R})|n(\mathbf{R} +\delta\mathbf{R})}) \approx -\mathbf{A}_{n}(\mathbf{R}) \cdot \delta\mathbf{R}
\end{equation*} which means that
\begin{multline}
 \label{chap_2:geombargmanninv}\gamma(C) = \oint_{C}{\mathbf{A}_{n}(\mathbf{R}) \cdot \delta\mathbf{R}}  = -\oint_{C}{\arg(\braket{n(\mathbf{R})|n(\mathbf{R} +\delta\mathbf{R})})} \\ = -\lim_{N\rightarrow\infty} \arg(\prod_{j=0}^{N-1}{\braket{n(\mathbf{R}(t + j\Delta t))|n(\mathbf{R}(t + (j+1)\Delta t))}})
\end{multline} where $\Delta t = \frac{T}{N} \ni \mathbf{R}(0)=\mathbf{R}(T)$.
Immediately we recognize that the RHS of Eq. \eqref{chap_2:geombargmanninv} is the argument of Bargmann invariant of states lying on the path $C$.

The Berry curvature surface integral can also be expressed in terms of the Bargmann invariant. We derive the relationship below for a 2-dimensional parameter space
labeled by $\bm{\lambda} = (\lambda_x, \lambda_y)$. This result can be easily generalized to arbitrary dimensions. Consider an infinitesimal square $Q$ in the parameter space 
consisting of points $(\lambda_x, \lambda_y), (\lambda_x + \delta\lambda_x, \lambda_y), (\lambda_x + \delta\lambda_x, \lambda_y + \delta\lambda_y), (\lambda_x, \lambda_y+\delta\lambda_y)$
in anti-clockwise order. The integral of berry connection around this closed loop is
\begin{multline*}
  \oint_{Q}{\mathbf{A}_{n}(\bm{\lambda})\cdot \delta\bm{\lambda}} = \\
  -\arg(\braket{n(\bm{\lambda})|n(\bm{\lambda} + \delta\lambda_x \hat{\mathbf{x}})}\braket{n(\bm{\lambda} + \delta\lambda_x \hat{\mathbf{x}})|n(\bm{\lambda} + \delta\lambda_x \hat{\mathbf{x}} + \delta\lambda_y \hat{\mathbf{y}})}\\ \braket{n(\bm{\lambda} + \delta\lambda_x \hat{\mathbf{x}} + \delta\lambda_y \hat{\mathbf{y}})|n(\bm{\lambda} + \delta\lambda_y \hat{\mathbf{y}})}\braket{n(\bm{\lambda} + \delta\lambda_y \hat{\mathbf{y}})|n(\bm{\lambda})})
\end{multline*}
Using stokes theorem
\begin{align*}
  \oint_{Q}{\mathbf{A}_{n}(\bm{\lambda})\cdot \delta\bm{\lambda}} &= \int_{Q}{\mathbf{B}_{n}(\bm{\lambda})\cdot d\mathbf{S}_{\bm{\lambda}}}\\
  &= \mathbf{B}_{n}(\bm{\lambda}) \delta\lambda_x\delta\lambda_y
\end{align*}
Combining these two results, we infer that\cite{rasta2016geometry}
\begin{multline}
 \mathbf{B}_{n}(\bm{\lambda}) \delta\lambda_x\delta\lambda_y = \\
 -\arg(\braket{n(\bm{\lambda})|n(\bm{\lambda} + \delta\lambda_x \hat{\mathbf{x}})}\braket{n(\bm{\lambda} + \delta\lambda_x \hat{\mathbf{x}})|n(\bm{\lambda} + \delta\lambda_x \hat{\mathbf{x}} + \delta\lambda_y \hat{\mathbf{y}})}\\ \braket{n(\bm{\lambda} + \delta\lambda_x \hat{\mathbf{x}} + \delta\lambda_y \hat{\mathbf{y}})|n(\bm{\lambda} + \delta\lambda_y \hat{\mathbf{y}})}\braket{n(\bm{\lambda} + \delta\lambda_y \hat{\mathbf{y}})|n(\bm{\lambda})})
\end{multline}
This result is particularly useful when numerically evaluating the surface integral of a berry phase in discretized parameter space.

\section{Non-abelian Berry connection}
In the derivation of Berry's phase, we assumed that there is/are no degeneracy or level crossings in the spectrum. In this section, we shall lift this restriction and examine its
consequences. If we assume that the ground state is $N$ fold degenerate, and begin in one of the degenerate states, after varying the parameters in closed loop, we examine the effect 
of this evolution on the state in the adiabatic limit \cite{wilczek1984appearance}. Unlike the previous discussion, the state can land up in a linear combination of the N-degenerate states.

The treatment of such systems require us to generalize the definition of a berry connection to
\begin{equation}
 (A_{i})_{ab}(\mathbf{R}) = i \braket{n_{b} | \frac{\partial}{\partial R_{i}} | n_{a}}
\end{equation} where $\ket{n_{a}}$ such that $a=1\dots N$, are the degenerate eigenstates of the hamiltonian. Notice that $\mathbf{A}$ is a d-dimensional (parameter space dimensions) vector of which
each individual component is a $N \times N$ matrix. 
Non-abelian Berry curvature is defined as 
\begin{equation}
 F_{ij} = \frac{\partial A_{i}}{\partial R_{j}} - \frac{\partial A_{j}}{\partial R_{i}} - i[A_{i}, A_{j}]
\end{equation} However, the Berry curvature is not a gauge invariant quantity. Instead, $tr(F_{ij})$ is gauge invariant.

As $\mathbf{A}$'s at different points in parameter space may not commute, we require the use of Path ordering operator (similar to Dyson series) to write the geometric phase 
as the integral of the berry connection
\begin{equation}
 U = \mathcal{P}\exp\left(\oint A_{i}dR_{i}\right)
\end{equation}
where \begin{equation}
       \ket{\psi_{a}}(t) = U_{ab}(t) \ket{n_{b}}(t)
      \end{equation}
We began with $\ket{\psi_{a}(0)} = \ket{n_{a}(0)}$.

\section{Chern numbers}
An important theorem of differential geometry, called the Gauss-Bonnet theorem states that the surface integral of Gaussian curvature over a closed surface is $2\pi$ times the
Euler characteristic of the surface \cite{rasta2016geometry,moore2014introduction}. Euler characteristic of a polyhedra is defined as $\chi = V - E + F$, where $V$ is the number of vertices, $E$ is the number of edges and $F$ is the
number of faces. The Euler characteristic for a continuous surface can be obtained by triangulation of the surface. The Euler characteristic of a closed surface is equal to 
$2(1-g)$, $g$ is the genus of the surface or intuitively the number of holes in the surface. Two surfaces that are continuously deformable into each other have the same genus.
Therefore, $\chi$ for a closed surface is an integral topological invariant and from Gauss-Bonnet theorem, the integral of the curvature over a closed surface is an integral multiple of
$2\pi$.
\begin{equation}
\iint_{S}{\mathbf{F}.d\mathbf{S}} = 2\pi\chi(S) = 4\pi(1-g(S)) = 2\pi C
\end{equation}

The Berry curvature satisfies the Gauss-Bonnet theorem, thereby the surface integral over a closed parameter space is an integral multiple of $2\pi$. The integers are called the
Chern numbers of the first class\cite{simon1983holonomy}. We shall see that the geometric phase and the chern numbers are closely related to hall conductivity in Chapter \ref{ch:qhe}.